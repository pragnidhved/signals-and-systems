\documentclass[12pt]{article}
\usepackage{amsmath,amsfonts,amssymb,float}
\title{NCERT Discrete}
\author{Pragnidhved Reddy\\\\EE23BTECH11050}
\date{}
\parindent 0px
\begin{document}
\maketitle
\textbf{Question 11.9.3.18:}\\
 Find the sum to n terms of the sequence $8,88,888,8888$\ldots\\
 \textbf{Solution :}\\
 \begin{table}[H]
 \centering
 \begin{tabular}{|c|c|c|c|}\hline
 $x_1$ & $x_2$ & $x_3$ & $x_4$\\ \hline
 8 & 88 & 888 & 8888\\ \hline
 \end{tabular}
  \caption{given inputs}
 \end{table}
 By this observation we can conclude that $$x_n=88\ldots n times$$
 This can also be represented as 
 \begin{equation}
 \tag{1}
  x_n=8(10)^0+8(10)^1+\ldots+8(10)^{n-1}
  \end{equation}
  Now, finding the sum of the series till $n$ terms:
 $$S_n=x_1+x_2+x_3+\ldots+x_n$$ 
 On substituting $(1)$ in the above equation we get 
\begin{equation}
\tag{2}
 S_n=n\times 8(10)^0+(n-1)\times 8(10)^1\ldots+1\times 8(10)^{n-1}
 \end{equation} 
 This is an AGP. Therefore,
 \begin{equation}
\tag{3}
 10S_n=n\times 8(10)^1+(n-1)\times 8(10)^2+\ldots+1\times 8(10)^n
 \end{equation}
 Now, subtracting $(2)$ from $(3)$
 \begin{align*}
 9S_n&=8(10)^1+8(10)^2+\ldots+8(10)^n-8n\\[10pt]
 S_n&=\left(\frac{8}{9}\right)\left(\left(\frac{10^n-1}{10-1}\right)10-n\right)\\[10pt]
 S_n&=\left(\frac{8}{81}\right)(10^{n+1}-9n-10)
 \end{align*}
 \end{document}

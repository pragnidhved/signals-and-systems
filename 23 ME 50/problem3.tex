\documentclass[journal,12pt,twocolumn]{IEEEtran}
\usepackage{amsmath,amsfonts,amssymb,float,gvv,graphicx,enumitem,array,esint}
\bibliographystyle{IEEEtran}
\vspace{3cm}
\title{NCERT Discrete}
\author{Pragnidhved Reddy\\EE23BTECH11050}
\date{}
\parindent 0px
\begin{document}
\maketitle
\newpage
\bigskip
\textbf{Question GATE 23 ME 50:}\\
The initial value problem
$\frac{dy}{dt}+2y=0, y(0)=1 $
is solved numerically using the forward Euler's method with a constant and positive time step of $\delta $.\\
Let $y_n$ represent the numerical solution obtained after $n$ steps. The condition $\abs{y_{n+1}} \leq \abs{y_n}$is satisfied if and only if $\delta$ does not exceed\\
\solution \\
By forward Euler's method formula 
\begin{align}
\label{eq:eq1}
    y(n+1)=y(n)+\delta  f(x,y)
\end{align}
From question we get
\begin{align}
\label{eq:eq2}
    \frac{dy}{dx}=-2y=f(x,y)
\end{align}
From \eqref{eq:eq2} in \eqref{eq:eq1}
\begin{align}
    y(n+1)&=y(n)+\delta(-2y(n))\\
    y(n+1)&=y(n)(1-2\delta )\\
    |y(n+1)|&=|y(n)||1-2\delta | \leq|y(n)|  \\
    |1-2\delta | &\leq 1\\
    \implies 0 \leq \delta  &\leq 1
\end{align}
From this we can say that the maximum value of $\delta  $ is 1
\end{document}

